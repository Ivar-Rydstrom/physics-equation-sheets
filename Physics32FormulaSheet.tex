\documentclass[fleqn]{article}

\title{\vspace{1mm}Physics 32}
\date{}

\usepackage[utf8]{inputenc} % not sure what this does
\usepackage[letterpaper, margin=3mm]{geometry} % paper format
\usepackage{mathtools} % lots of stuff
\usepackage{amssymb} % some more obscure symbols

\usepackage{graphicx}
\usepackage{caption}
\usepackage{subcaption}
\usepackage{float} % [H] support
\usepackage{amsmath,amssymb,amsfonts,cancel}
\usepackage{algorithmic}
\usepackage{graphicx}
\usepackage{textcomp}
\usepackage{xcolor}
\pagecolor{white}
\setlength{\mathindent}{1pt}
\usepackage{multicol}
\setlength{\columnsep}{1mm}
\usepackage{color}
\setlength{\columnseprule}{1pt}
\def\columnseprulecolor{\color{black}}
%uncomment above so you can see if you're going into the next column (if you care)

\begin{document}

\newcommand{\myint}[4]{\int_{#1}^{#2} \!  #3  \, \mathrm{d} #4}
\newcommand{\paren}[1]{\!\left({#1}\right)}
\newcommand{\commutator}[2]{\left[ #1\, , #2\right]}
\newcommand{\expec}[1]{\! \left\langle {#1}\right\rangle}
\newcommand{\expect}[1]{\! \langle {#1}\rangle}
\newcommand{\mysum}[3]{\sum_{#1}^{#2} \, #3 }
\newcommand{\pdiv}[3]{ \frac{\partial^{#3}{#1}}{\partial{#2}^{#3} } }
\newcommand{\deriv}[3]{ \frac{\mathrm{d}^{#3}{#1}}{\mathrm{d}{#2}^{#3} } }
\newcommand{\braket}[1]{\!\left| #1 \right\rangle}
\newcommand{\half}[0]{\frac{1}{2}}
%some commands I made, for derivatives you can leave #3 blank for first order


\section*{\centering \underline{Physics 32}}
\begin{multicols*}{3} %three columns
\subsection*{ \centering \underline{Gravity \& Kepler's Laws}:}
\[ F_g = G \frac{M_a M_b}{r^2}, \qquad U_g = -G \frac{M_a M_b}{r}\]
\[ G = 6.67x10^{-11} \]
\raggedright
    {\textbf{First Law:} Elliptical orbits\linebreak
    \textbf{Second Law:} Equal areas swept over equal times (conservation of angular momentum)\linebreak
    \textbf{Third Law:} Law of Periods: \( T^2 \propto R^3 \)
    }
%\columnbreak %these send you into the next column, in case formatting is an issue

\subsection*{ \centering \underline{Oscillations}:}
\[ F_k = -kX \qquad U_k = \frac{1}{2}kx^2 \]
\[ x(t) = x_{max}\cos{(\omega t + \Phi)} \]
\[ \deriv{x}{t}{2} = -\omega^2x \qquad T = \frac{2\pi}{\omega_0} \]
\[ \textbf{Spring:} \quad T = 2\pi \sqrt{\frac{m}{k}} \]
\[ \textbf{Pendulum:} \quad T = 2\pi\sqrt{\frac{\ell}{g}} \]
\[ x_{damped}(t) = x_{max}e^{-bt/2m}\cos{(\omega^\prime + \Phi)} \]
\[ \omega^\prime = \sqrt{\frac{k}{m} - \frac{b^2}{4m^2}} \]

\subsection*{ \centering \underline{Fluid Dynamics}:}
\[ \textbf{Pascal's Principle:} \; \Delta P = \frac{F}{A}, \quad\frac{F_1}{A_1} = \frac{F_2}{A_2} \]
\[ P_2 - P_1 = \rho g(h_2 - h_1) \Rightarrow dP = \rho gdh \]
\[ F_{buoyant} = m_{fluid}g = \rho_{fluid}V_{displaced}g \]
\[ \textbf{Continuity Eq:} \quad A_1V_1 = A_2V_2 = \frac{Vol}{t} \]
\[ \textbf{Bernoulli's Eq:} \quad P + \rho gh + \frac{1}{2}\rho V^2 = Const\]

\subsection*{ \centering \underline{Waves}:}
\[ v = \lambda f, \quad T = \frac{1}{f}, \quad \omega = \frac{2\pi}{T}, \quad k = \frac{2\pi}{\lambda} \]
\[ V = \frac{\lambda}{T} = \lambda f = \boxed{\frac{\omega}{k}} \quad Secret \; Equation!\]
\[ v_{string} = \sqrt{\frac{F_T}{\mu}}, \quad F_T = Tension, \quad \mu = \frac{m}{\ell} \]
\[ v_{fluid} = \sqrt{\frac{Bs}{\rho}}, \quad v_{solid} = \sqrt{\frac{E}{\rho}} \]
\(E/B\) = Elastic/Bulk Moduli.\linebreak
\[ E_{wave} = 2\pi^2 f^2 \rho (vtArea) {x_{max}}^2 \]
\[ I_{intensity} = \frac{Power}{Area} \]
\[ y_{traveling}(x,t) = y_{max}\sin{(kx \mp \omega t)} \]
(Minus for +x, plus for -x propagation)
\[ \textbf{Wave Equation:} \quad \pdiv{y}{x}{2} = \frac{1}{v^2}\left(\pdiv{y}{t}{2}\right) \]

\subsection*{ \centering \underline{Sound}: }
\[ f_{beat} = |f_1-f_2| \]
\[ \beta = 10\log{\left(\frac{I}{I_0}\right)}, \qquad I_0 = 10^{-12} \frac{w}{m^2} \]
\[ \boxed{\beta = 20\log{\left(\frac{y_{max}}{I_0}\right)}} \quad Secret \; Equation! \]
\[ f_{observer} = f_{source}\left(\frac{v_{sound}\pm v_{obs}}{v_{sound}\mp v_{source}}\right) \]
\[ Pick \quad +v_{obs} = Towards, \quad -v_{obs} = Away \]

\subsection*{ \centering \underline{Light and Optics}: }
\[ Reflection: \quad \theta_i = \theta_f \]
\[ Refraction \, (Snell's): \; n_0\sin{\theta_0} = n_f\sin{\theta_f} \]
\[ \textbf{Double Slit:} \quad d\sin{\theta} = m\lambda \]
\[ m_{bright} = 1, 2, 3..., \quad m_{dark} = \frac{1}{2}, \frac{3}{2}, \frac{5}{2}... \]
\[ (Use \; small \; angle \; approx: \quad \tan{\theta} = \frac{y}{D}) \]
\[ 2d = (m + \frac{1}{2})\lambda^\prime + (m + \frac{1}{2})\frac{\lambda}{n^\prime} \]

\end{multicols*}

\end{document}