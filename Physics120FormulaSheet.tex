\documentclass[fleqn]{article}

\title{\vspace{1mm}Physics 120}
\date{}

\usepackage[utf8]{inputenc} % not sure what this does
\usepackage[letterpaper, margin=3mm]{geometry} % paper format
\usepackage{mathtools} % lots of stuff
\usepackage{amssymb} % some more obscure symbols

\usepackage{graphicx}
\usepackage{caption}
\usepackage{subcaption}
\usepackage{float} % [H] support
\usepackage{amsmath,amssymb,amsfonts,cancel}
\usepackage{algorithmic}
\usepackage{graphicx}
\usepackage{textcomp}
\usepackage{xcolor}
\pagecolor{white}
\setlength{\mathindent}{1pt}
\usepackage{multicol}
\setlength{\columnsep}{1mm}
\usepackage{color}
%\setlength{\columnseprule}{1pt}
%\def\columnseprulecolor{\color{black}}
%uncomment above so you can see if you're going into the next column (if you care)

\begin{document}

\newcommand{\myint}[4]{\int_{#1}^{#2} \!  #3  \, \mathrm{d} #4}
\newcommand{\paren}[1]{\!\left({#1}\right)}
\newcommand{\commutator}[2]{\left[ #1\, , #2\right]}
\newcommand{\expec}[1]{\! \left\langle {#1}\right\rangle}
\newcommand{\expect}[1]{\! \langle {#1}\rangle}
\newcommand{\mysum}[3]{\sum_{#1}^{#2} \, #3 }
\newcommand{\pdiv}[3]{ \frac{\partial^{#3}{#1}}{\partial{#2}^{#3} } }
\newcommand{\deriv}[3]{ \frac{\mathrm{d}^{#3}{#1}}{\mathrm{d}{#2}^{#3} } }
\newcommand{\braket}[1]{\!\left| #1 \right\rangle}
\newcommand{\half}[0]{\frac{1}{2}}
%some commands I made, for derivatives you can leave #3 blank for first order


\section*{\centering \underline{Physics 120}}
\begin{multicols*}{3} %three columns
\subsection*{ \centering \underline{Equations of State}:}















\end{multicols*}

\end{document}